\documentclass[11pt]{article}

\usepackage{geometry}
\usepackage{hyperref}
\usepackage{array} % Required for m{...} column type

% Set the page margins
\geometry{letterpaper, margin=1in}

% Set up hyperlinks
\hypersetup{
    colorlinks=true,
    linkcolor=blue,
    filecolor=magenta,      
    urlcolor=cyan,
}

\title{CS 370: Undergraduate Reading and Research}
\author{}
\date{Spring 2024}

\begin{document}

\maketitle

\section*{General Information}
\textbf{Student:} Jose Alvarez Loredo, \textbf{EID:} jea3268
\href{mailto:enriprivacc@utexas.edu}{enriprivacc@utexas.edu} \\
\textbf{Supervising Instructor:} Siddhartha Chatterjee, \href{mailto:sidchat@utexas.edu}{sidchat@utexas.edu} \\
**Adjustments to the course format are not final.

\section*{Description}
370/350C Cossover project: Evaluate the space-time tradeoffs for different hashing algorithms on modern computing systems, measuring architectural effects especially for very large tables.

\section*{Class Meetings:} Schedule to be announced (TBA). Additionally, to ensure effective collaboration, we can organize brief meetings aligned with the CS 350C class schedule in the spring semester, either before or after the class sessions.

\section*{Initial Information}

\paragraph{Recommended Paper:} \href{https://arxiv.org/pdf/2209.07663.pdf}{arXiv:2209.07663}.

\paragraph{Hash Tables:}
There are three primary types of hash tables to consider, categorized by their collision resolution methods:
Open Addressing, Chaining, Cuckoo Hashing.

\paragraph{Starting Points:}
\begin{itemize}
    \item A good starting point for understanding cuckoo hashing is the Wikipedia entry: \href{https://en.wikipedia.org/wiki/Cuckoo_hashing}{Cuckoo Hashing on Wikipedia}.
    
    \item For practical insights and techniques, the following blog has several pertinent articles: \href{http://www.idryman.org/blog/2017/05/03/writing-a-damn-fast-hash-table-with-tiny-memory-footprints/}{Writing a Damn Fast Hash Table with Tiny Memory Footprints}.
\end{itemize}

\paragraph{Performance Measurements:}
It is noted that many performance metrics available are outdated. Hence, it is recommended to prioritize performance measurements, beginning with the most efficient implementations currently available.

\clearpage

\clearpage

\begin{table}[h!]
    \centering
        \begin{tabular}{ | m{3cm} | m{10cm}| m{2cm} | }
        \hline
        \textbf{Topic} & \textbf{Description} & \textbf{ Due Date} \\
        \hline
        Preliminary Work &
            \textbf{Review Readings:} Carefully read through the materials provided in the \emph{'Starting Points'} section.

            \textbf{Additional Readings:} Engage in an extensive review of supplementary academic papers relevant to the topic.
    
            \textbf{Outline Completion:} Based on your understanding from the readings, complete the course outline for CS-370. Cover all necessary topics and key concepts that will be explored throughout the course.
        & 01/16/2024 \\

        \hline
        Hashing & 
            \textbf{Implementation Task:} Gain a deeper insight into hash table operations by implementing the Open Addressing, Chaining, and Cuckoo Hashing methods.
            
            \textbf{Note:} Implementations are for educational purposes only. The performance measurements for the project will utilize the optimized algorithms developed by external experts.
        & 01/19/2024 \\

        \hline
        Research Paper & 
            \textbf{Research Paper Template:} Develop a LaTeX template for the research paper, adhering to standard formatting rules. Investigate additional guidelines on structuring the paper, ensuring compliance with common academic standards and best practices.

            \textbf{Paper:} Finish the Introduction section
        & 01/26/2024 \\

        \hline
        Literature Review &
            \textbf{Literature Review Preparation:} Conclude all supplementary readings pertinent to the research paper. Proceed with drafting the literature review section, integrating insights and findings from these readings.
        & 02/02/2024 \\

        \hline
        Checkpoint 1 & 
            \textbf{Instructor Consultation:} Schedule a progress check meeting with the supervising instructor. Actively incorporate any feedback or suggestions received during the consultation into your work.

            \textbf{Paper:} Finish the Literature Review section
        & 02/09/2024 \\

        \hline
        Methodology & 
        \textbf{Simulation/Implementation:} Simulate the optimized hashing algorithms. Ensure the chosen approach effectively demonstrates the algorithms' functionalities and nuances.

        \textbf{Testing Environment Setup:} Establish a stable and consistent testing environment that mirrors modern computing systems. Critical for obtaining accurate performance metrics of the algorithms.
        & 02/16/2024 \\
        \hline
        
        \hline
        Data Preparation & 
            \textbf{Data Set Preparation:} Acquire or create extensive data sets necessary for evaluating the hash tables. The data sets should be representative of real-world scenarios to ensure the relevance and applicability of the test results.
            
            \textbf{Paper:} Finish the Methodology section 
        & 02/23/2024 \\
        \hline

\end{tabular}
\caption{Project Timeline}
\label{tab:timeline}
\end{table}
\begin{table}[h!]
    \centering
        \begin{tabular}{ | m{3cm} | m{10cm}| m{2cm} | }
        \hline
        \textbf{Topic} & \textbf{Description} & \textbf{ Due Date} \\

        \hline
        Experimentation & 
            \textbf{Run Tests:} Execute the hashing algorithms using the prepared data sets. Ensure diverse testing conditions are employed to effectively assess the algorithms' performance under various architectural impacts.
    
            \textbf{Data Collection:} Meticulously document the performance metrics of each algorithm across different test conditions.
        & 03/01/2024 \\

        \hline
         Results &
            \textbf{Data Presentation:} Devise an effective method for incorporating the collected data into the research paper's results section. Focus on clarity and comprehensibility in presenting the findings.
    
            \textbf{Continued Testing:} If necessary, continue executing tests throughout the week to gather additional data or validate previous results.
            
            \textbf{Paper Revision:} Initiate a comprehensive review and revision process for the entire research paper. This stage includes refining content, enhancing clarity, and ensuring coherence across all sections.
        & 03/08/2024 \\

        \hline
        Catch-up week & 
        \textbf{March 11 - 16:	Spring Break}
        & 03/15/2024 \\

        \hline
        Checkpoint 2 & 
            \textbf{Instructor Consultation:} Schedule a progress check meeting with the supervising instructor. Actively incorporate any feedback or suggestions received during the consultation into your work.
            
            \textbf{Paper:} Finish the Results section
        & 03/22/2024 \\

        \hline
        Analysis &
            \textbf{Algorithm Comparison:} Conduct a detailed analysis of each hashing algorithm, focusing on their performance in the context of space-time tradeoffs. Assess how efficiently each algorithm manages memory and processing time.
            
            \textbf{Data Visualization:} Utilize graphs or charts to effectively illustrate the space-time trade-offs. Ensure that these visualizations are clear, accurate, and aid in understanding the comparative analysis.
        & 03/29/2024 \\

        \hline
        Discussion & 
            \textbf{Architectural Impact Assessment:} Examine the effects of various computing architectures on the performance of each algorithm. This evaluation should highlight how different hardware configurations influence algorithm efficiency.
                    
            \textbf{Paper:} Finish the Discussion section
        & 04/05/2024 \\

        \hline
        Final Research &
        \textbf{Conclusion Development:} Begin crafting the conclusion section of the paper. Concurrently, engage in post-results research to deepen your understanding of the observed outcomes. Aim to draw meaningful connections between the results and the broader context of your research topic.
        & 04/12/2024 \\

        \hline
        Conclusion & 
            \textbf{Paper:} Finish the Conclusion section
        & 04/19/2024 \\

        \hline
        Wrap-up & 
            \textbf{End of Course:} All tasks and assignments pertaining to CS-370 should be fully completed by this time.
            
            \textbf{Paper:} Finish Revisions
        & 04/26/2024 \\
        
        \hline
        End & 
            \textbf{Final Class Day}
        & 04/29/2024 \\
        \hline
        % Add more rows as needed
        \end{tabular}
    \caption{Project Timeline}
    \label{tab:timeline}
\end{table}

\end{document}